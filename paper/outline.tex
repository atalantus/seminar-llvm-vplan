\documentclass[sigplan,11pt,nonacm]{acmart}
\settopmatter{printfolios}

\usepackage{booktabs} % For formal tables
\usepackage{subcaption}
\usepackage{tikz}
\usepackage{pgfplots}
\usepackage{pgfplotstable}
\usepackage{hyphenat}
\usepackage{todonotes}
\usepackage[babel]{csquotes}


\begin{document}
\title{Utilizing Parallel Workers: \\LLVM's Vectorization Plan}
\author{Jonas Fritsch}
\affiliation{%
  \institution{Technical University of Munich}
}
\email{jonas.fritsch@tum.de}

\begin{abstract}
Lorem ipsum.
\end{abstract}

\maketitle

%%%%%%%%%%%%%%%%%%%%%%%%%%%%%%%%%%%%%%%%%%%%%%%%%%%%%%%%%%%%%%%%%%%%%%%%%%%%%%%%%%%%%%%%%%%%%%%%%%%%%%


\section{Introduction}
\label{sec:introduction}

\begin{enumerate}
  \item Modern CPUs (SIMD-Registers, SIMD ISAs)
  \item SIMD performance gains
  \item Auto-Vectorization to help programmers utilize SIMD functionality out of the box 
  (explicit vectorization as time consuming especially when supporting multiple architectures)
  \item LLVM as popular, platform-agnostic compiler framework~\cite{10.5555/977395.977673}
  \item auto-vectorization complexity $\rightarrow$ wish for well designed, 
  interoperable system (VPlan) $\rightarrow$ transition from Loop Vectorizer + SLP Vectorizer to 
  Vectorization Plan $\rightarrow$ ongoing 8 year long effort started by Intel
  \item rough paper outline
\end{enumerate}


%%%%%%%%%%%%%%%%%%%%%%%%%%%%%%%%%%%%%%%%%%%%%%%%%%%%%%%%%%%%%%%%%%%%%%%%%%%%%%%%%%%%%%%%%%%%%%%%%%%%%%


\section{Background}
\label{sec:background}
\begin{enumerate}
  \item Vectorization explanation (with Code\hyp{}Examples)
  \begin{itemize}
    \item Loop-Level Vectorization (Inner / Outer-Loops~\cite{llvmextloopvec,intelouterloop,10.5555/2337159.2337210,llvmouterloop,10.1145/1454115.1454119})
    \item Function Vectorization~\cite{llvmextloopvec}
    \item Superword-Level Parallelism\\Vectorization~\cite{10.1145/349299.349320}~(maybe already \cite{10.1145/3519939.3523701})
  \end{itemize}
  \item Vectorization Constraints~\cite{llvmveccontrolflow,intelvecessen} (reuse code\hyp{}examples from above)
  \begin{itemize}
    \item not always legal / applicable, e.g.: backwards data-dependencies, pointer aliasing, 
    platform backwards-compatability / support, control flow complexity (function calls, register pressure), FP inaccuracies (\texttt{-ffast-math})
    \item possible performance loss, e.g.: costly conversions (horizontal aggregations, integer division) resulting in slower overall code
    on average
    \item larger code size, e.g.: need for scalar loop-tail / epilogue
    \item (potential security vulnerablities~\cite{2023arXiv230201131K})
  \end{itemize}
\end{enumerate}


%%%%%%%%%%%%%%%%%%%%%%%%%%%%%%%%%%%%%%%%%%%%%%%%%%%%%%%%%%%%%%%%%%%%%%%%%%%%%%%%%%%%%%%%%%%%%%%%%%%%%%


\section{LLVM's Vectorization Plan}
\label{sec:vplan}
  Overview~\cite{llvmvec,llvmvplan,llvmextloopvec,llvmintrvplan,llvmvplanstate,llvmvplanrv,llvmouterloop,llvmvplanupdate}
\begin{enumerate}
  \item Phase 1: Legality
  \item Phase 2: Planning (Vectorization Plans)
  \begin{enumerate}
    \item Building initial VPlan based on Phase 1 Constraints (Generate one or more up-to-date Code Examples for VPlan similar to~\cite{llvmintrvplan,llvmvplanstate,llvmouterloop})
    $\rightarrow$ Explain different components (e.g.: \texttt{VPBasicBlock}, \texttt{VPRegionBlock}, etc.) and
    correlation between generated VPlan structure, original scalar code. Show some general optimizations.
    \item Modelling Cost. Not yet part of VPlan. (related work? already adopted?~\cite{POHL2020102106})
    \item Optimizing/Transforming VPlans~\cite{llvmvplanupdate}
  \end{enumerate}
  \item Phase 3: Execution - Materialize best VPlan
\end{enumerate}


%%%%%%%%%%%%%%%%%%%%%%%%%%%%%%%%%%%%%%%%%%%%%%%%%%%%%%%%%%%%%%%%%%%%%%%%%%%%%%%%%%%%%%%%%%%%%%%%%%%%%%


\section{Related Work}
\label{sec:relatedwork}
\begin{enumerate}
  \item Other auto-vectorization algorithms / possible improvements~\cite{10.5555/3615924.3615932,4336219,llvmveccontrolflow,saarlandrv,10.1145/3303117.3306172}
  \\Some of those should already be (partially) implemented in LLVM
  \item Auto-Vectorization in GCC~\cite{autovecoptgcc, vecoptgcc}
  \item Comparison with GCC~\cite{gccllvmveccomp,10.1155/2021/3264624}
  \item (Polyhedral compilation techniques~\cite{llvmpolyhedral})
  \item (LLM-based Vectorization~\cite{taneja2024llmvectorizerllmbasedverifiedloop})
\end{enumerate}


%%%%%%%%%%%%%%%%%%%%%%%%%%%%%%%%%%%%%%%%%%%%%%%%%%%%%%%%%%%%%%%%%%%%%%%%%%%%%%%%%%%%%%%%%%%%%%%%%%%%%%


\section{Summary and Future Work}
\label{sec:summary}
\begin{enumerate}
  \item VPlan Summary~\cite{llvmvplan,llvmvplanupdate}
  \item VPlan Future Roadmap~\cite{llvmvplan,llvmvplanupdate} (e.g.: combining outer-/inner-loop paths, full function vectorization, etc.)
  \item (Comparison/Outlook auto-vectorization and explicit vectorization~\cite{10.1155/2021/3264624, 9802745, DBLP:conf/vldb/BensonER23})
\end{enumerate}


%%%%%%%%%%%%%%%%%%%%%%%%%%%%%%%%%%%%%%%%%%%%%%%%%%%%%%%%%%%%%%%%%%%%%%%%%%%%%%%%%%%%%%%%%%%%%%%%%%%%%%

\bibliographystyle{ACM-Reference-Format}
\bibliography{paper} % read paper.bib file

\end{document}
